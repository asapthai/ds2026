\documentclass[11pt]{article}
\usepackage[utf8]{inputenc}
\usepackage{geometry}
\usepackage{graphicx}
\usepackage{hyperref}
\usepackage{listings}
\geometry{margin=1in}

\title{Practical Work 1: TCP File Transfer}
\author{Name: Vu Xuan Thai\\\\Student id: 23BI14397}
\date{\today}

\begin{document}
\maketitle

\section*{Goal}
Implement a 1-1 file transfer over TCP/IP in CLI based on a socket interface.

\section{Design of the protocol}
We use a simple binary protocol with a fixed header followed by raw file bytes:
\begin{itemize}
  \item 4 bytes: filename length (unsigned 32-bit integer, network byte order)
  \item File name (UTF-8), length given by previous field
  \item 8 bytes: file size in bytes (unsigned 64-bit integer, network byte order)
  \item File content: raw bytes of length specified by file size
\end{itemize}

This keeps parsing deterministic and simple.  
After the transfer, the server replies with a short ASCII status: \texttt{OK}, \texttt{INCOMPLETE}, or \texttt{ERROR}.

\section{System organization}
The system is implemented as two CLI programs:
\begin{itemize}
  \item \texttt{server.c} -- listens for TCP connections and receives files into a specified directory.
  \item \texttt{client.c} -- connects to the server and sends a file using the protocol above.
\end{itemize}

\subsection*{Sequence diagram}

\begin{verbatim}
Client  -> Server: connect
Client  -> Server: header (filename_length, filename, filesize)
Client  -> Server: file bytes...
Server  -> Client: "OK" / "INCOMPLETE" / "ERROR"
Client  -> Server: close
\end{verbatim}

\section{Important Code Snippets}

\subsection*{Sending header (client)}
\begin{verbatim}
uint32_t name_len_net = htonl(strlen(filename));
send(sock, &name_len_net, 4, 0);
send(sock, filename, strlen(filename), 0);
uint64_t size_net = htonll(filesize);
send(sock, &size_net, 8, 0);
\end{verbatim}

\subsection*{Receiving header (server)}
\begin{verbatim}
recv_all(client_fd, &fname_len_net, 4);
fname_len = ntohl(fname_len_net);

recv_all(client_fd, filename_buf, fname_len);
filename_buf[fname_len] = '\0';

recv_all(client_fd, &size_net, 8);
filesize = ntohll(size_net);
\end{verbatim}

\section{Who does what}
\begin{itemize}
  \item Member 1: Implement server, test receiving small/large files, write report.
  \item Member 2: Implement client, test with various files, prepare demonstration.
\end{itemize}

\section{How to run the programs}

\subsection*{Server}
\begin{verbatim}
./server --host 0.0.0.0 --port 9000 --outdir received_files
\end{verbatim}

\subsection*{Client}
\begin{verbatim}
./client --host SERVER_IP --port 9000 --file path/to/file
\end{verbatim}

\end{document}
