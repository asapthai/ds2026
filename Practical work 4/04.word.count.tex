\documentclass{article}
\usepackage[utf8]{inputenc}
\usepackage{listings}
\usepackage{xcolor}

\title{Practical Work 4: Word Count}
\author{Name: Vu Xuan Thai \\ ID: 23BI14397}
\date{\today}
\begin{document}
\maketitle
\section{Introduction}
This report describes the implementation of the Word Count algorithm using a custom C++ MapReduce framework. The goal is to count the frequency of each word in a dataset by utilizing the Map and Reduce programming model.
\section{Implementation Choice}
We implemented the framework in C++ because it offers low-level control over memory and data structures. Since there is no standard MapReduce library for C++, we created a ``MapReduce'' base class that handles the core workflow: Mapping, Shuffling (Sorting), and Reducing.
\section{Architecture}
\subsection{The Map Phase}
The \texttt{Map} function takes a line of text as input. It uses a string stream to split the line into individual words. For every word found, it emits a key-value pair: \texttt{<word, 1>}.
\subsection{The Shuffle Phase}
After mapping, the framework sorts all intermediate pairs alphabetically by key. This groups identical words together (e.g., all ``apple'' keys sit next to each other).
\subsection{The Reduce Phase}
The \texttt{Reduce} function receives a word and a list of counts (e.g., \texttt{<"apple", [1, 1, 1]>}). It sums these counts to produce the final total frequency for that word.
\section{Conclusion}
The implementation successfully demonstrates the MapReduce logic. The separation of the \texttt{Map} and \texttt{Reduce} functions allows the code to be easily parallelized in a real distributed environment.
\end{document}